\documentclass[12pt]{article}
  \textheight=24cm
  \textwidth=16cm
  \oddsidemargin=0.5cm
  \headheight=0cm
  \topmargin=0cm
  \parskip .1cm \parsep .1cm
  \voffset=-2cm

\begin{document}


\begin{center}{\bf \Large cours 8 } 
\end{center}



\begin{center}{\bf \Large Compilateurs gfortran gcc  } \end{center}



Sous Unix, les compilateurs gfortran,gcc,g++ effectuent l'operation
de compilation et de link a la suite. \\
exemple: gfortran ou fc comme sur convex, \\
xlf sur ibm risc ... \\
syntaxe de la commande 

gfortran file.f -o file options lib.a otherfile.o


Le fichier qui est compile est file, on veut le linker avec
une librairie (collection de fichiers objet) notee lib.a sous unix.
Cette librairie peut avoir ete construite par l'utilisateur (commande
'ar') ou non. 

\section{Remarques importantes:}

\begin{itemize}
\item On a adjoint un autre fichier otherfile.o qui lui est
sous la forme objet, ceci montre les deux effets de gfortran: compilation
et link.

\item ATTENTION A l'ORDRE pour les bibliotheques. Si le
programme a compiler a besoin de routines dans lib.a alors
il faut placer lib.a APRES file.f \\
gfortran lib.a file.f -o file \\
donne une erreur alors que \\
gfortran file.f lib.a -o file \\
compile et lie correctement.

\item Le compilateur gfortran est aussi un compilateur C. Il est donc
possible de melanger du C et du fortran.

\end{itemize}

Exemples \\

gfortran -o exec file1.f file2.o file3.c lib.a


gfortran -c file1.f \\

gfortran -c file3.c \\
gfortran -o exec file1.o file2.o file3.o file4.a

{\bf Utilite de 'make'}

\section{Les options de gfortran}

Il y en a une multitude que l'on pourra consulter en faisant
'man gfortran' . Elles sont de la forme '-lettre' . Voici celles
qui sont le plus utiles.

\begin{itemize}


\item '-c' Effectue une compilation seule, sans link. La pratique Unix
  est d'effectuer d'abord cette operation sur tous les fichiers puis
  de creer l'executable avec la commande 'gfortran file.o file2.o...'.
  C'est la base du makefile (voir plus bas).
  Cette option permet aussi de verifier la syntaxe d'un code rapidement.

\item '-g' Etablit une table de symboles pour le 'debuggage' du code. Voir
  doc sur gdb le debuggeur de haut niveau sous unix. Attention cette
  option est en general incompatible avec '-O' qui est l'option d'optimisation.

\item '-l' Suivie d'un nom cette option d'edition de lien cherche la bibliothque
  libnom.a dans le repertoire indique par l'option -L puis dans /lib et
  /usr/lib. Cette option est utilisee en general pour les bibliotheques
  de l'utilisateur.

\item '-o' Elle est indiquee ci-dessus et permet de nommer l'executable.
  Ainsi le resultat de la compilation cidessus sera un executable nomme
  'file'.

\item '-O' Optimise le code. Cette option doit etre maniee avec precaution
   , et en particulier il est bon de comparer les resultats du code
   avec et sans optimisation. Sur les nouvelles machines -O1 -O2 ..
   Une des optimisations simple est le deroulement de boucles.

\item '-p' Cette option avancee permet d'etablir le pourcentage de temps
   passe dans chaque procedure. Voir amelioration du code.

\item '-u'  Exige une declaration explicite pour chaque variable utilisee
   dans le code. Tres pratique pour verifier la syntaxe avant compilation.

\end{itemize}

Un executable = gros fichier \\
multiples informations utilisables apres un arret errone (core
dump). \\
strip exe



Possible d'utiliser debugger gdb pour obtenir
des informations sur la cause de l'arret. 


\end{document}

